\documentclass[11pt]{article}

\usepackage{report}

\usepackage[utf8]{inputenc} % allow utf-8 input
\usepackage[italian]{babel}
\usepackage[T1]{fontenc}    % use 8-bit T1 fonts
\usepackage[colorlinks=true, linkcolor=black, citecolor=blue, urlcolor=blue]{hyperref}       % hyperlinks
\usepackage{url}            % simple URL typesetting
\usepackage{booktabs}       % professional-quality tables
\usepackage{amsfonts}       % blackboard math symbols
\usepackage{nicefrac}       % compact symbols for 1/2, etc.
\usepackage{microtype}      % microtypography
\usepackage{lipsum}		% Can be removed after putting your text content
\usepackage{graphicx}
\usepackage{natbib}
\usepackage{doi}
\setcitestyle{aysep={,}}

% tree structure
\usepackage{forest}
%%%%%%%%%


%%%%%%%%%%%%%% CODE

\usepackage{listings}
\usepackage{color} %red, green, blue, yellow, cyan, magenta, black, white
\definecolor{mygreen}{RGB}{28,172,0} % color values Red, Green, Blue
\definecolor{mylilas}{RGB}{170,55,241}

\lstset{language=Matlab,%
    %basicstyle=\color{red},
    breaklines=true,%
    morekeywords={matlab2tikz},
    keywordstyle=\color{blue},%
    morekeywords=[2]{1}, keywordstyle=[2]{\color{black}},
    identifierstyle=\color{black},%
    stringstyle=\color{mylilas},
    commentstyle=\color{mygreen},%
    showstringspaces=false,%without this there will be a symbol in the places where there is a space
    numbers=left,%
    numberstyle={\tiny \color{black}},% size of the numbers
    numbersep=9pt, % this defines how far the numbers are from the text
    emph=[1]{for,end,break},emphstyle=[1]\color{red}, %some words to emphasise
    %emph=[2]{word1,word2}, emphstyle=[2]{style},    
}

%----------------------------------------------------------------------------------------
%	PYTHON CODE THEMPLATE
%----------------------------------------------------------------------------------------
\usepackage{listings}
\usepackage{xcolor}

\definecolor{codegreen}{rgb}{0,0.6,0}
\definecolor{codegray}{rgb}{0.5,0.5,0.5}
\definecolor{codepurple}{rgb}{0.58,0,0.82}
\definecolor{backcolour}{rgb}{0.95,0.95,0.92}

\lstset{language=Matlab,%
     % backgroundcolor=\color{backcolour},
    commentstyle=\color{codegreen},
    keywordstyle=\color{magenta},
    numberstyle=\tiny\color{codegray},
    stringstyle=\color{codepurple},
    basicstyle=\ttfamily\footnotesize,
    breakatwhitespace=false,
    breaklines=true,
    captionpos=b,
    keepspaces=true,
    numbers=left,
    numbersep=5pt,
    showspaces=false,
    showstringspaces=false,
    showtabs=false,
    aboveskip=6mm,
    belowskip=6mm,
    tabsize=2   
}
%%%%%%%%%%%%%%

\title{Cholesky Analysis}

\author{
  \Large\textbf{Mario Avolio}\\
  \texttt{800995}
  \and
  \Large\textbf{Simone Benitozzi}\\
  \texttt{889407}
}

% Uncomment to remove the date
\date{\today}

% Uncomment to override  the `A preprint' in the header
\renewcommand{\headeright}{Cholesky Analysis - Metodi del calcolo Scientifico}
\renewcommand{\undertitle}{Metodi del calcolo Scientifico}
\renewcommand{\shorttitle}{}

%%% Add PDF metadata to help others organize their library
%%% Once the PDF is generated, you can check the metadata with
%%% $ pdfinfo template.pdf
% \hypersetup{
% pdftitle={A template for the arxiv style},
% pdfsubject={q-bio.NC, q-bio.QM},
% pdfauthor={David S.~Hippocampus, Elias D.~Striatum},
% pdfkeywords={First keyword, Second keyword, More},
% }

\begin{document}
\maketitle

\newpage
\tableofcontents
\thispagestyle{empty}

\newpage
\thispagestyle{empty}


\newpage
\setcounter{page}{1}
\section{Introduzione}
TODO: Introduzione

\begin{lstlisting}[language=Python, caption=Python Code Example]
import os

import cv2
import numpy as np

from src.costants import SPRITE_PATH

TESTS_PATH = os.path.dirname(__file__)
LOGS_PATH = os.path.join(TESTS_PATH, 'logs')
MAP_PATH = os.path.join(TESTS_PATH, 'map')

img_rgb = cv2.imread(os.path.join(MAP_PATH, "Matrix1.png"))
img = cv2.cvtColor(img_rgb, cv2.COLOR_BGR2GRAY)
template = cv2.imread(os.path.join(SPRITE_PATH, "yellow.png"), 0)

w, h = template.shape[::-1]

res = cv2.matchTemplate(img, template, cv2.TM_CCOEFF_NORMED)
threshold = 0.85
loc = np.where(res >= threshold)
for pt in zip(*loc[::-1]):
    cv2.rectangle(img_rgb, pt, (pt[0] + w, pt[1] + h), (0, 0, 255), 2)

cv2.imwrite(os.path.join(TESTS_PATH, 'res.png'), img_rgb)
\end{lstlisting}

\begin{lstlisting}[language=Matlab, caption=Matlab Code Example]
clear all, close all, clc

% --- matrix loading
folder_path = '';
matrix_name = 'ex15.mat';
matrix_path = strcat(folder_path, matrix_name);
load(matrix_path)
A = Problem.A;
% spy(A)
clear Problem matrix_path folder_path matrix_name

% memory usage before execution (after loading matrix)
before_mem = memory;

% problem parameters
xe = ones(size(A, 1), 1);
b = A*xe;

% --- system solving

tic %starts timing
% Cholesky decomposition
R = chol(A);

% solution computing
x = R\(R'\b);

time = toc; % ends timing
fprintf("Time elapsed: %f seconds\n", time)

% --- memory usage estimation
after_mem = memory;
mem_difference = after_mem.MemUsedMATLAB-before_mem.MemUsedMATLAB;
mem_difference_mb = mem_difference*1e-6;
fprintf("\nTotal memory used: %e bytes (%f MB)\n", mem_difference, mem_difference_mb)

% --- error estimation
err = norm(x - xe, 2) / norm(xe, 2);
fprintf("\nRelative error: %e\nepsilon: %e\n", err, eps(1))
\end{lstlisting}






\section{Descrizione del dominio di riferimento e obiettivi dell’elaborato}
\section{Scelte di design}
Per gestire al meglio la giusta separazione tra gli elementi del progetto si è deciso di sfruttare un particolare pattern strutturale definito dallo schema sottostante. Il modello, cattura il comportamento dell'applicazione in termini di dominio del problema e gestisce direttamente i dati, la logica e le regole del progetto. 


\begin{forest}
  for tree={
    font=\ttfamily,
    grow'=0,
    child anchor=west,
    parent anchor=south,
    anchor=west,
    calign=first,
    edge path={
      \noexpand\path [draw, \forestoption{edge}]
      (!u.south west) +(7.5pt,0) |- node[fill,inner sep=1.25pt] {} (.child anchor)\forestoption{edge label};
    },
    before typesetting nodes={
      if n=1
        {insert before={[,phantom]}}
        {}
    },
    fit=band,
    before computing xy={l=15pt},
  }
[Customer Personality Analysis Project
  [ML-Porject.Rproj]
  [Data
    [marketing\_campaign.csv]
  ]
  [DOC
    [Presentation
    [...]]
    [Report
    [...]]
  ]
  [Script
  [D-TREE.R]
  [DescriptionOfData.R]
  [EDA.R]
  [K-MEANS.R]
  [PCA.R]
  [D-TREE-Model Evaluation.R]
  [Functions
  [Functions.R]]
  ]
  [Output
  [Plots
  [...]]
  [Data
  [...]]]
  [Other
  [...]]
  [README.MD]
]
\end{forest}\\

\section{Descrizione dei Dati}








% BIBLIOGRAPHY
\bibliographystyle{unsrtnat}
\bibliography{references}  %%% Uncomment this line and comment out the ``thebibliography'' section below to use the external .bib file (using bibtex) .


%%% Uncomment this section and comment out the \bibliography{references} line above to use inline references.
% \begin{thebibliography}{1}

% 	\bibitem{kour2014real}
% 	George Kour and Raid Saabne.
% 	\newblock Real-time segmentation of on-line handwritten arabic script.
% 	\newblock In {\em Frontiers in Handwriting Recognition (ICFHR), 2014 14th
% 			International Conference on}, pages 417--422. IEEE, 2014.

% 	\bibitem{kour2014fast}
% 	George Kour and Raid Saabne.
% 	\newblock Fast classification of handwritten on-line arabic characters.
% 	\newblock In {\em Soft Computing and Pattern Recognition (SoCPaR), 2014 6th
% 			International Conference of}, pages 312--318. IEEE, 2014.

% 	\bibitem{hadash2018estimate}
% 	Guy Hadash, Einat Kermany, Boaz Carmeli, Ofer Lavi, George Kour, and Alon
% 	Jacovi.
% 	\newblock Estimate and replace: A novel approach to integrating deep neural
% 	networks with existing applications.
% 	\newblock {\em arXiv preprint arXiv:1804.09028}, 2018.

% \end{thebibliography}


\end{document}
