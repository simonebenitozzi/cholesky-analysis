\section{Introduzione}

L'obiettivo dell'elaborato è principalmente quello di confrontare l'ambiente open-source con quello proprietary-software per la risoluzione di un problema comune, in maniera tale da trarne i giusti spunti e considerazioni.

Nello specifico verrà analizzata l'implementazione e il funzionamento in ambienti di programmazione e Sistemi Operativi differenti del metodo di Cholesky, per la risoluzione sistemi lineari per matrici sparse, simmetriche e definite positive.

In quanto agli ambienti sono stati confrontati MATLAB, linguaggio chiuso e utilizzabile sotto licenza, e Python, open source e ricco di librerie per la risoluzione di una gran vastità di problemi.
Per quanto riguarda invece i Sistemi Operativi, il confronto vedrà la contrapposizione tra esecuzioni in Windows e Linux.

Lo scopo è quello di presentare un'analisi quanto più approfondita possibile, che tenga conto dei possibili trade-off e non si limiti solamente all'aspetto tecnico-informatico, ragionando quindi in termini di performance ed efficacia, ma che abbracci anche l'ambito manageriale, considerando i costi e le ripercussioni di una scelta rispetto all'altra, simulando un contesto reale di decision making e facendo sì che lo studio possa portare ad una scelta ponderata.